\def\I{\relax} % needed to trick Django’s templating system
\setupbodyfont[ss,8pt]
\language[{{language}}]
\setuplayout[header=10mm,footer=10mm,topspace=0mm]
\setupTABLE[column][width=1cm,height=1cm,style=bold]
%\setupheader[location=high]
\setuppagenumbering[state=stop,location=high]
\definecolor[blocked][s=.8]
\definecolor[puzzlebackground][s=1.0]
\definecolor[puzzleforeground][c=1.0,m=0,y=0,k=0] % number
\definecolor[puzzleframe][s=0.0]
\definecolor[solutionbackground][s=0.0]
\definecolor[solutionforeground][s=1.0]
\definecolor[solutionframe][s=1.0]
\definecolumnset[twocols][n=2]
\setupitemize[n,packed]

\starttext

\setupTABLE[{{ block.0 }}][{{ block.1 }}][background=color,backgroundcolor=blocked]

\subject{\I  }
{
\bTABLE[background=color,backgroundcolor=puzzlebackground,color=puzzleforeground,framecolor=puzzleframe]
\bTR\bTD {{ cell.1 }}\eTD\eTR
\eTABLE
}

\blank[big]

\startcolumnset[twocols]

\subsubject{\I  }

\startitemize[n]
\sym{\I {{qu.num}}} {{ qu.text }}
\stopitemize

\column
\subsubject{\I  }

\startitemize[n]
\sym{\I {{qu.num}}} {{ qu.text }}
\stopitemize

\stopcolumnset

%\setupTABLE[column][width=1em,height=1em,align={middle,lohi}]
\subject{\I  }
{
\setupTABLE[column][width=2em,height=2em,align={middle,lohi}]
\bTABLE[background=color,backgroundcolor=solutionbackground,color=solutionforeground,framecolor=solutionframe]
\bTR\bTD {{ cell.0 }}\eTD\eTR
\eTABLE

}

\stoptext